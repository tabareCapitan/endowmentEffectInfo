
\title{\vspace{-3cm}
      Expecting to get it: \\ An endowment effect for information
      }

% ADD LINK TO UPTODATE VERSION AND SUPPLEMENTARY+REPLICATION MATERIAL

\author{Tabaré Capitán
          \thanks{Corresponding author: \href{mailto:Tabare.Capitan@gmail.com}{\texttt{Tabare.Capitan@gmail.com}}. For helpful comments or thoughtful discussions, we thank Hunt Allcott, Alec Brandon, James Heckman, Stephan Kroll, John List, Daniel Millinet, and Stephen Newbold. We also thank participants in seminars at UA Anchorage, the University of Chicago, Colorado State University, the 2019 North American Economic Science Association meetings, and the 2020 Bogotá Experimental Economics Conference. Financial support from USDA/NIFA (grant number 2016-09907) is gratefully acknowledged. The collection of data for this study was approved by the IRB at University of Wyoming.}
        \and
        Linda Thunström
        \and
        Klaas van ‘t Veld
          \\ \small{Department of Economics, University of Wyoming}
        \and
        Jonas Nordström
          \\ \small{Lund University and University of Copenhagen}
        }

\date{\today \\\href{https://www.tabarecapitan.com/jmp/}{\small{\textcolor{red}{(Click here for the most recent version)}}}}

\maketitle

\thispagestyle{empty}   % Suppress page number (must be under \maketitle)

\begin{abstract}

\noindent
In this paper we introduce the \enquote{endowment effect for information}: that the value of information is higher when information is expected than when information is not expected. We demonstrate such an endowment effect for information in an experiment in which we manipulate expectations. Furthermore, we show that this result is consistent with predictions from two (standard and expanded) leading theories of expectations-based reference-dependent preferences. Our endowment effect for information implies that net benefits from information policies may vary with people’s expectations (e.g., consumers who regularly see calorie information on menus might value the information more because they expect it).
\\
\\
\textit{Keywords:} expectations, reference-dependent preferences, preferences for information, information avoidance, endowment effect, welfare analysis.
\\
\textit{JEL classification:} D01, D80, D83, D84, D91, C91.

\end{abstract}

\clearpage

\pagenumbering{arabic} % Start numbering in page 2 from #1


% First version
% We show, both theoretically and empirically, that an endowment effect for information arises from the expectation of getting information. Furthermore, we show that this endowment effect for information matters to welfare analysis of information policies (e.g., support for information policies may depend on the timing of the assessment), and can be put to advantage in situations where widespread dissemination of information is socially desirable (e.g., environmental cost of different activities or benefits of using masks during the COVID-19 global pandemic).
