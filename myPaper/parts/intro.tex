% CONCRETE EXAMPLE -------------------------------------------------------------

Ana and Beto, separately, have decided to go to a café to eat a chocolate cake. Both are equally concerned about their calorie intake and have the same belief about the calorie content of the cake, but Ana expects the café to have calorie information on the menu and Beto does not. In this paper we show, both theoretically and empirically, that Ana’s expectation makes the information more valuable to her than to Beto.  In other words, an endowment effect for information arises from the very expectation of getting the information. We are not aware of previous work in which this endowment effect has been identified, let alone tested.\footnote{However, at least two previous studies provide suggestive evidence of an endowment effect for information. First,  \citet{cawleyImpactInformationDisclosure2020} find that participants with high exposure to calorie information are more supportive of calorie labels. Second, \cite{nordstromStrategicIgnoranceHealth2020} find that participants that have been informed about the calorie content in a meal are more likely to state that they would have used the calorie information, compared to participants that could choose to be informed or not.}

% INFORMATION VS RISK ----------------------------------------------------------

% COMMENTS

% KLAAS:  the parenthetical bits in (third sentence) read *too* terse to me now. The reader won’t understand the mapping from risk to information, and you don’t want to lose her 1.5 paragraphs into the paper.
% -> Expanded "a non-degenerate lottery defined by a prior distribution of beliefs ABOUT THE INFORMATION"


The endowment effect we observe is closely related to the endowment effect for risk predicted by the theory of \citet{koszegiReferenceDependentRiskAttitudes2007} and tested by \citet{sprengerEndowmentEffectRisk2015}. Their endowment effect for risk arises because those expecting to face risk (a non-degenerate lottery over consumption outcomes) are more tolerant to risk than those expecting to face an outcome that is certain. In our case, expecting to receive information is akin to expecting to face risk (a non-degenerate lottery defined by a prior distribution of beliefs about the information), whereas \emph{not} expecting to receive information is akin to expecting an outcome that is certain (the mathematical expectation of the prior distribution of beliefs about the information).\footnote{\citet{koszegiReferenceDependentConsumptionPlans2009} develop another theory of expectations-based reference-dependent preferences\textemdash\emph{news utility}\textemdash that might seem closer to our work. However, this theory is about reference-dependence with respect to the \emph{content} of the information (i.e., what you might find out). Instead, the endowment effect for information arises from reference-dependence with respect to information \emph{delivery} (i.e., whether you find out), regardless of the content.}

% PONDERING WHAT'S BEST TO START THIS PARAGRAPH:
% Yet the parallel to an endowment effect for risk is not perfect.
% But the parallel to an endowment effect for risk is not perfect.
% However, the parallel to an endowment effect for risk is not perfect.
% The parallel to an endowment effect for risk is not perfect, however. (Klaas)
% Even so, the parallel to an endowment effect for risk is not perfect.

However, the parallel to an endowment effect for risk is not perfect. Because the role of information in decision making goes beyond mere updating of beliefs, preferences for risk are not the same as preferences for information. Instrumental information can be used to adjust behavior (e.g., calorie information can induce Ana or Beto to eat less or more of their cake, or exercise more or less strenuously afterwards), and even non-instrumental information can simply spark curiosity \citep{loewensteinPsychologyCuriosityReview1994,sharotHowPeopleDecide2020}.

% THEORY -----------------------------------------------------------------------

Our endowment effect for information fits in the theoretical literature on expectations-based reference-dependent preferences \citep{marzilliericsonEndowmentEffect2014,odonoghueChapterReferenceDependentPreferences2018}.\footnote{\citet{sprengerEndowmentEffectRisk2015} reviews the related empirical literature, concluding that \enquote{Though both positive and negative results have been documented, an account of the literature would indicate some promise for the relevance of expectations in rationalizing reference-dependent behavior} (p1458).} In this literature, utility is separated into intrinsic utility (i.e., utility from outcomes) and gain-loss utility (i.e., utility from comparing outcomes to a referent). While the intrinsic utility is typically modeled in the same way across different theories, the specification of the gain-loss utility differs between the two leading theoretical approaches. The Disappointment Aversion approach \citep{bellDisappointmentDecisionMaking1985,loomesDisappointmentDynamicConsistency1986,gulTheoryDisappointmentAversion1991} collapses the reference lottery into a single value and measures the gain-loss utility relative to that value. In contrast, the  Kőszegi and Rabin approach \citep{koszegiModelReferenceDependentPreferences2006,koszegiReferenceDependentRiskAttitudes2007} measures the gain-loss utility from every element of the outcome lottery relative to each element of the reference lottery separately, and then aggregates these gain-loss utilities to a single value.\footnote{Consequently, the distinction between these two theoretical approaches is consequential only when the referent is stochastic, i.e., a reference lottery.}  In this paper, we show that both approaches predict an endowment effect for instrumental information, but only the Kőszegi and Rabin approach predicts an endowment effect for non-instrumental information.

% EXPERIMENTAL EVIDENCE --------------------------------------------------------

% -> INTRO TO LAB EXPERIMENT

To test our theoretical prediction of an endowment effect for \emph{instrumental} information, we conducted a laboratory experiment in which we offered participants a free cake, manipulated their expectations about whether they will get information of the cake’s calorie content, and asked them whether they wanted to receive the information. Our empirical results are consistent with the theoretical prediction.\footnote{As opposed to the case of non-instrumental information, the theoretical prediction from both theoretical approaches is the same for the case of instrumental information---existence of an endowment effect for information. Since our study is not about distinguishing between the two theories, we chose to use instrumental information to test for the endowment effect. As such, our results can only be interpreted as evidence consistent with the endowment effect, but do not help to distinguish the empirical validity of the two theories of expectations-based reference-dependent preferences. To further ensure the instrumentality of the information, we excluded from the analysis 29 observations from participants who stated they would not eat the cake.} The idea of manipulating expectations to investigate the endowment effect builds on \possessivecitet{koszegiModelReferenceDependentPreferences2006} definition of a person’s referent as not the status quo, but rather \enquote{the \emph{expectations} a person held in the \emph{recent} past}, or more specifically \enquote{her probabilistic beliefs about the relevant consumption outcome held between the time she first focused on the decision determining the outcome and shortly before consumption occurs} (p.1141). Hence, the status quo can be viewed as a special case of this more general definition: in most experiments that treat the status quo as the referent, subjects plausibly expect the status quo to continue in the future. More details of the experimental design follow.

% -> CONTROL HETEROGENEITY OF EXPECTATIONS

Participants came to the laboratory with their own expectations regarding the availability of calorie information (i.e., whether they would get it or not).  To exercise some control over the potential heterogeneity in these expectations before our key experimental manipulation, we started the experiment asking participants to choose their preferred dessert from eleven consecutive hypothetical menus \emph{without} calorie information. The intuition is that by providing participants with \emph{experience} making dessert choices without calorie information, we shift the mass of the distribution of expectations towards not expecting to get information. In other words, we test the endowment effect for information with participants who (mostly) do not expect to receive information.\footnote{In this first step we manipulate the referent (i.e., the expectation to receive information), we discuss the potential interaction between this manipulation and the key experimental manipulation in the discussion of the experimental design.}

% -> KEY EXPERIMENTAL MANIPULATION

In the key experimental manipulation, all participants are asked if they want to learn the calorie content of three desserts on a menu, one of which, a cake, is marked as theirs to receive. For half of the participants, the menu includes a calorie-information column with the actual calorie numbers “temporarily” xxx-ed out; for the other half, the menu shows nothing calorie-related. In other words, we vary participants’ first-focus on calorie information to vary their sense of endowment. The idea behind this manipulation builds on the first-focus intuition implied by \possessivecitet{koszegiModelReferenceDependentPreferences2006} definition of the referent, which \citet{sprengerEndowmentEffectRisk2015} also exploits to test the endowment effect for risk.\footnote{Sprenger notes that the intuition is \enquote{in line with the psychological literature on \enquote{cognitive reference points} (Rosch 1974) and decision anchoring} (p.1462), which has shown referents to be susceptible to experimental variation.} Specifically, the \emph{xxx}-ed out calorie column on the first group's menu is intended to make participants focus first on getting calorie information when considering their choice whether to actually see it, and in that sense feel \emph{more endowed} with the information. Conversely, the absence of any calorie-related information on the second group's menu is intended to make participants focus first on not getting information when considering the same choice, and in that sense feel \emph{less endowed} with the information.\footnote{In a typical endowment effect study, the endowment is a physical object such as a mug, a chocolate bar, or money. As such, most studies in this literature label their experimental groups as \enquote{not endowed} and \enquote{endowed}. This is non-controversial because it is hard to argue that the act of giving the object to a participant is not the same as endowing said participant with the object. However, in our case the referent is an expectation instead of an object. We do not claim that our manipulation \emph{fully} sets the referent. Instead, we claim that our experimental manipulations vary the \emph{degree} to which participants feel endowed with the information. Consequently, following \citet{heffetzEndowmentEffectExpectations2014}, we chose to more precisely label our experimental groups as \emph{more endowed} and \emph{less endowed}.}

% RESULTS OF LOW BASELINE

In line with our theoretical prediction, our empirical results support the existence of an endowment effect for instrumental information. Fewer participants in the \emph{less endowed} group (52\%) chose to receive information about the calorie content of the cake than in the \emph{more endowed} group (71\%). This difference of 19 percentage points is both economically and statistically significant (pvalue=0.02 from a t-test and pvalue=0.05 from a Fisher’s exact test, both one sided). This is the main empirical result of this paper.

% INTRO TO EFFECT REDUCED BY EXPERIENCE

So far, we have provided empirical evidence consistent with the theoretical prediction of an endowment effect for information. We now turn to test whether this effect is reduced by experience, which is an empirical result that has been consistently found for the (standard) endowment effect for goods \citep{listDoesMarketExperience2003, engelmannReconsideringEffectMarket2010, listDoesMarketExperience2011}. To do so, we replicated the key experimental manipulation discussed above with one important difference: we started the experiment asking participants to choose their preferred dessert from eleven consecutive hypothetical menus \emph{with} calorie information (instead of menus without calorie information, as we did before).  In this case, the intuition is that by providing participants with \emph{experience} making dessert choices with calorie information, we shift the mass of the distribution of expectations towards expecting to get information. In other words, we now test the endowment effect for information with participants who (mostly) do expect to receive information. This priming manipulation is motivated by \possessivecitet{koszegiModelReferenceDependentPreferences2006} rationalization of \possessivecitet{listDoesMarketExperience2003} empirical finding that market experience reduces the (standard) endowment effect. The argument is that, because experienced traders are more likely to expect to trade their acquisitions, the endowment effect for them is smaller. More specifically, because experienced traders are less likely to extrapolate the status quo of having owned an item in the recent past to an expectation of retaining the item in the future, they experience less loss aversion when parting with the item through trading it away. As a result, they behave more like they would if they never owned the item in the first place, and in that sense exhibit less of an endowment effect.\footnote{List shows that market experience reduces the endowment effect in two related papers \citep{listDoesMarketExperience2003, listDoesMarketExperience2011}.
The key issue to identify the effect of market experience on the endowment effect is that market experience is endogenous (i.e., experienced traders might be experienced precisely because of their lower endowment effect). In his first paper \cite{listDoesMarketExperience2003} \enquote{attempts to econometrically parse treatment (market experience) from selection, (but) his results rely on his modeling assumptions} (p.314), and in his second paper \cite{listDoesMarketExperience2011} \enquote{attempts to rectify this issue (i.e., relying on modeling assumptions) by making market experience exogenous} (p. 314). Although \citet{koszegiModelReferenceDependentPreferences2006} rationalize the empirical finding as presented in the first paper \citep{listDoesMarketExperience2003}, our experimental design is closer to \citep{listDoesMarketExperience2011} because we also use a two-stage experiment in which first manipulate experience (using menus with calorie information) and then manipulate the endowment (or expectation to receive information). Hence, we are able to identify the endowment effect for information separately for unexperienced and experienced participants, without relying on modeling assumptions.}

% KR'S RATIONALIZATION OF LIST'S APPLIED TO US

Our manipulation applies the argument in the opposite direction. Whereas List’s experienced traders already expected to \emph{give up} items in their transaction, through their frequent exposure to trading, our participants with experience making choices from menus with calorie information should already expect to \emph{gain} information in the key experimental manipulation, through their frequent exposure to seeing it on menus (at the beginning of the experiment). These participants, we therefore predict, should be less likely to extrapolate the status quo of \emph{not} having information to an expectation of not getting it in future, and thus experience \emph{more} loss aversion when giving up the information (i.e., their referent is more likely to be getting the information). As a result, they will behave more like they would if they felt endowed with the information in the first place, and in that sense exhibit less of an endowment effect.\footnote{Note that the endowment \enquote{effect} is defined as a difference in behavior (e.g., willingness to trade a mug for a pen) depending on one’s sense of endowment. List’s experienced traders behaved similarly whether or not they were endowed with an item, thus exhibiting no endowment effect. We predict that participants with experience making choices from menus with calorie information will also behave similarly whether or not they are endowed with information, thus again exhibiting no endowment effect.}

% RESULTS HIGH BASELINE

As expected, this time we find a smaller endowment effect for information. About 60\% of the participants in the \emph{less endowed} group and about 65\% of the participants in the \emph{more endowed} chose to receive calorie information about the cake. This difference of 6 percentage points is not statistically significant (pvalue=0.27 from a t-test and pvalue=0.34 from a Fisher’s exact test, both one sided). In summary, we interpret this result as suggestive evidence that the endowment effect for information is similar to the endowment effect for goods in that both are reduced by experience.

% POLICY IMPLICATIONS ----------------------------------------------------------

The endowment effect for information has implications for welfare analysis of policies that mandate, encourage, or restrict, the delivery of various kinds of information. Examples include the Right-to-Know acts and eco-label programs (e.g., USDA organic and Energy Star) in environmental regulation; GMO, nutritional, and calorie labeling in agricultural and food regulation; graphic warnings on cigarette packaging, disclosure of medical risks and side effects, and diagnostic testing in health regulation; SEC financial disclosure (e.g., Sarbanes-Oxley) and mortgage disclosure in financial regulation; and ethical labels (e.g., animal cruelty and child labor), financial disclosure by politicians, and conflict-mineral content in other areas. These information policies are ubiquitous and often subject to cost-benefit analysis (CBA) requirements.

Under the traditional assumption (in economics) that the value of information is always non-negative \citep{stiglerEconomicsInformation1961}, CBA is straightforward in principle: the only hurdle that needs to be checked is that estimated information benefits outweigh (often low) implementation costs. But recent research (reviewed by \citet{golmanInformationAvoidance2017}) has challenged the traditional assumption, noting that information can evoke negative feelings such as anxiety or guilt, so that even costless information can harm people (i.e., the value of information can be negative).\footnote{For example, if information evokes negative emotions or feelings (e.g., anxiety, fear, guilt, regret, annoyance, or pain), these negative emotions or feelings might lead to information avoidance \citep{koszegiHealthAnxietyPatient2003,danaExploitingMoralWiggle2007,karlssonOstrichEffectSelective2009,sweenyInformationAvoidanceWho2010,osterOptimalExpectationsLimited2013,grossmanStrategicIgnoranceRobustness2014,hertwigHomoIgnoransDeliberately2016,onwezenWhenIndifferenceAmbivalence2016,savolainenApproachingAffectiveBarriers2016,thunstromStrategicSelfignorance2016,grossmanSelfImageWillfulIgnorance2017,damgaardHiddenCostsNudging2018,thunstromEndogenousAttentionCosts2019,sunsteinRuiningPopcornWelfare2019}.} The possibility to do harm with information policies complicates welfare analysis because\textemdash even ignoring the cost of implementation\textemdash their welfare effects can range from negative to positive \citep{damgaardHiddenCostsNudging2018, allcottWelfareEffectsNudges2019,buteraDeadweightLossSocial2019,rafiqHowMuchCalorie2019,thunstromWelfareEffectsNudges2020}.\footnote{The complication is greater if there is heterogeneity in negative feelings \citep{sunsteinRuiningPopcornWelfare2019}, or in the ability to respond to information \citep{robertsNudgeProofDistributiveJustice2018}}

Our findings point to further complications. Most immediately, the existence of an endowment effect for information implies that the outcome of welfare analysis may depend on timing. Consider, for example, a cost-benefit analysis of a policy that requires cafés to provide calorie information on their menus. In the presence of experience effects like those identified in our experiment, this analysis would be more likely to conclude in favor of the policy if conducted after the policy’s introduction, when people have come to expect the information. More generally, the existence of an endowment effect for information implies that the role of the referent, including potential heterogeneity of referents across consumers, ought to be explicitly acknowledged and accounted for in welfare analysis.

     % XXXXXXXX    PARAGRAPH BELOW NEEDS SOME WORK     XXXXXXXXXXXXXX %

On the positive side, the endowment effect for information may provide a new policy tool to reduce avoidance of \enquote{good} information, such as information about health and financial risks. The goal of this policy tool would be to raise people's expectation to receive information and, via the endowment effect, thereby increase their valuation of that information (leading to less information avoidance). For example, during the COVID-19 pandemic, people might be less likely to ignore information about the risks of engaging in various activities if they expect to get such information.

We return to the discussion of welfare implications in the concluding section, after first introducing our theoretical framework and empirical evidence in the next sections.
