% CONCRETE EXAMPLE -------------------------------------------------------------

Ana and Beto, separately, have decided to go to a café to eat a chocolate cake. Both are equally concerned about their calorie intake and have the same belief about the calorie content of the cake, but Ana expects the café to have calorie information on the menu and Beto does not. In this paper we show, both theoretically and empirically, that Ana’s expectation makes the information more valuable to her than to Beto.  In other words, an endowment effect for information arises from the very expectation of getting the information. We are not aware of previous work in which this endowment effect has been identified, let alone tested.\footnote{However, at least two previous studies provide suggestive evidence of an endowment effect for information. First, in a study by \citet{cawleyImpactInformationDisclosure2020} participants with high exposure to calorie information were more supportive of calorie labels, and \cite{nordstromStrategicIgnoranceHealth2020} find that participants that have been informed about the calorie content in a meal are more likely to state that they would have used the calorie information, compared to participants that could choose to be informed or not.}

% INFORMATION VS RISK ----------------------------------------------------------

The endowment effect we observe is closely related to the endowment effect for risk predicted by the theory of \citet{koszegiReferenceDependentRiskAttitudes2007} and tested by \citet{sprengerEndowmentEffectRisk2015}. Their endowment effect for risk arises because those expecting to face risk (a non-degenerate lottery over consumption outcomes) are more tolerant to risk than those expecting to face a certain outcome. In our case, expecting to receive information is akin to expecting to face risk (a non-degenerate lottery defined by a prior distribution of beliefs), whereas \emph{not} expecting to receive information is akin to expecting a certain outcome (the mathematical expectation of the prior distribution of beliefs).\footnote{\citet{koszegiReferenceDependentConsumptionPlans2009} develop another theory of expectations-based reference-dependent preferences\textemdash\emph{news utility}\textemdash that might seem closer to our work. However, their \emph{news utility} theory is about reference-dependence with respect to the \emph{content} of the information (i.e., what you might find out). Instead, the endowment effect for information arises from reference-dependence with respect to information \emph{delivery} (i.e., whether you find out), regardless of the content.}

% PONDERING WHAT'S BEST TO START THIS PARAGRAPH:
% Yet the parallel to an endowment effect for risk is not perfect.
% But the parallel to an endowment effect for risk is not perfect.
% However, the parallel to an endowment effect for risk is not perfect.
% The parallel to an endowment effect for risk is not perfect, however. (Klaas)
% Even so, the parallel to an endowment effect for risk is not perfect.

However, the parallel to an endowment effect for risk is not perfect. Because the role of information in decision making goes beyond mere updating of beliefs, preferences for risk are not the same as preferences for information. Instrumental information can be used to adjust behavior (e.g., calorie information can induce Ana or Beto to eat less or more of their cake, or exercise more or less strenuously afterwards), and even non-instrumental information can simply spark curiosity \citep{loewensteinPsychologyCuriosityReview1994,sharotHowPeopleDecide2020}.

% THEORY -----------------------------------------------------------------------

Our endowment effect for information fits in the theoretical literature on expectations-based reference-dependent preferences \citep{marzilliericsonEndowmentEffect2014,odonoghueChapterReferenceDependentPreferences2018}.\footnote{\citet{sprengerEndowmentEffectRisk2015} reviews the related empirical literature, concluding that \enquote{Though both positive and negative results have been documented, an account of the literature would indicate some promise for the relevance of expectations in rationalizing reference-dependent behavior} (p1458).} In this literature, utility is separated into intrinsic utility (i.e., utility from outcomes) and gain-loss utility (i.e., utility from comparing outcomes to a referent). While the intrinsic utility is typically modeled in the same way across different theories, the specification of the gain-loss utility differs between the two leading theoretical approaches. The Disappointment Aversion approach \citep{bellDisappointmentDecisionMaking1985,loomesDisappointmentDynamicConsistency1986,gulTheoryDisappointmentAversion1991} collapses the reference lottery into a single value and measures the gain-loss utility relative to that value. In contrast, the  Kőszegi and Rabin approach \citep{koszegiModelReferenceDependentPreferences2006,koszegiReferenceDependentRiskAttitudes2007} measures the gain-loss utility from every element of the outcome lottery relative to each element of the reference lottery separately, and then aggregates these gain-loss utilities to a single value.\footnote{Consequently, the distinction between these two theoretical approaches is consequential only when the referent is stochastic, i.e., a reference lottery.}  In this paper, we show that both approaches predict an endowment effect for instrumental information, but only the Kőszegi and Rabin approach predicts an endowment effect for non-instrumental information.

% EXPERIMENTAL EVIDENCE --------------------------------------------------------

In addition, we find evidence consistent with an endowment effect for information in a two-stage laboratory experiment, in which we offer participants a free cake and manipulate their expectations about whether they will get information about the cake’s calorie content. The idea of manipulating expectations to investigate the endowment effect builds on \citet{koszegiModelReferenceDependentPreferences2006}'s definition of a person’s referent as not the status quo, but rather \enquote{the \emph{expectations} a person held in the \emph{recent} past}, or more specifically \enquote{her probabilistic beliefs about the relevant consumption outcome held between the time she first focused on the decision determining the outcome and shortly before consumption occurs} (p.1141). Hence, the status quo can be viewed as a special case of this more general definition: in most experiments that treat the status quo as the referent, subjects plausibly expect the status quo to continue in the future.

In the first stage, we vary participants’ \emph{experience} with calorie information to vary their sense of endowment. All participants are asked to choose their preferred dessert from eleven consecutive different menus, but those menus show calorie information for only half of the participants, and no calorie information for the other half. This priming manipulation is motivated by \citet{koszegiModelReferenceDependentPreferences2006}'s rationalization of \citet{listDoesMarketExperience2003}'s empirical finding that market experience reduces the (standard) endowment effect. The argument is that, because experienced traders are more likely to expect to trade their acquisitions, the endowment effect for them is smaller. More specifically, because experienced traders are less likely to extrapolate the status quo of having owned an item in the recent past to an expectation of retaining the item in the future, they experience less loss aversion when parting with the item through trading it away. As a result, they behave more like they would if they never owned the item in the first place, and in that sense exhibit less of an endowment effect.\footnote{\citet{koszegiModelReferenceDependentPreferences2006} rationalize \citet{listDoesMarketExperience2003}'s empirical finding, but our experiment is closer to \citet{listDoesMarketExperience2011}'s experimental design because we use a two-stage experiment in which we first manipulate participant's experience and then manipulate their endowment to test for the endowment effect. Both of List's papers reach the same conclusion, but his early work relied on modeling assumptions (instead of exogenous manipulation) to address the potential endogeneity of market experience (i.e., experience traders might be experienced precisely because of their lower endowment effect).}

% unexperience: 51% -> 70%
% experience:   60% -> 65%

Our manipulation applies the argument in the opposite direction. Whereas List’s experienced traders already expected to \emph{give up} items in their transaction, through their frequent exposure to trading, our experienced participants (in the second stage) should already expect to \emph{gain} information, through their frequent exposure to seeing it on menus (in the first stage). Experienced participants, we therefore predict, should be less likely to extrapolate the status quo of \emph{not} having information to an expectation of not getting it in future, and thus experience \emph{more} loss aversion when giving up the information (i.e., their referent is more likely to be getting the information). As a result, they will behave more like they would if they felt endowed with the information in the first place, and in that sense exhibit less of an endowment effect.\footnote{Note that the endowment \enquote{effect} is defined as a difference in behavior (e.g., willingness to trade a mug for a pen) depending on one’s sense of endowment. List’s experienced traders behaved similarly whether or not they were endowed with an item, thus exhibiting no endowment effect. We predict that experienced participants will also behave similarly whether or not they are endowed with information, thus again exhibiting no endowment effect.}

In the second stage, we vary participants' \emph{first-focus} on calorie information to vary their sense of endowment. All participants are asked if they want to learn the calorie content of three desserts on a menu, one of which, a cake, is marked as theirs to receive. For half of the participants, the menu includes a calorie-information column with the actual calorie numbers \enquote{temporarily} \emph{xxx}-ed out; for the other half, the menu shows nothing calorie-related. This is the key manipulation of our experiment. It builds on the first-focus intuition implied by \citet{koszegiModelReferenceDependentPreferences2006}'s definition of the referent, which \citet{sprengerEndowmentEffectRisk2015} also exploits to test the endowment effect for risk.\footnote{Sprenger notes that the intuition is \enquote{in line with the psychological literature on \enquote{cognitive reference points} (Rosch 1974) and decision anchoring} (p.1462), which has shown referents to be susceptible to experimental variation.} Specifically, the \emph{xxx}-ed out calorie column on the first group's menu is intended to make participants focus first on getting calorie information when considering their choice whether to actually see it, and in that sense feel \emph{endowed} with the information. Conversely, the absence of any calorie-related information on the second group's menu is intended to make participants focus first on not getting information when considering the same choice, and in that sense feel \emph{not endowed} with the information.\footnote{Although we label our participants as \emph{endowed} or \emph{not endowed}, we do not claim that our manipulation \emph{fully} sets the referent. Instead, our experimental manipulations vary the \emph{degree} to which participants feel endowed with the information. A more precise way to label our participats, used by \citet{heffetzEndowmentEffectExpectations2014}, would be to use \emph{more endowed} and \emph{less endowed}. Nevertheless, we favor our labels to be consistent with the language typically used in the literature of the endowment effect.}

Our results support the existence of an endowment effect for information, in line with our theoretical prediction: among participants not primed in the first stage to expect calorie information, those endowed with information in the second stage asked to keep the information significantly more often than non-endowed participants asked to add it. Moreover, consistent with the argument that experience with information matters, this endowment effect was much weaker, and not statistically significant, for participants primed in the first stage to expect information.

% POLICY IMPLICATIONS ----------------------------------------------------------

This endowment effect has implications for welfare analysis of policies that mandate, encourage, or restrict, the delivery of various kinds of information. For example: the Right-to-know acts and Eco-label programs (e.g., USDA organic and Energy Start), in environmental regulation; GMO, nutritional, and calorie labeling, in agricultural and food regulation; graphic warnings on cigarette packaging, disclosure of medical risks and side effects, and diagnostic testing, on health regulation; SEC financial disclosure (e.g., Sarbanes-Oxley) and mortgage disclosure, on financial regulation; and ethical labels (e.g., animal cruelty and child labor), financial disclosure by politicians, and conflict-mineral content, in other areas. These information policies are ubiquitous and often subject to Cost-Benefit Analysis (CBA) requirements.

Under the traditional assumption (in economics) that the value of information is always non-negative \citep{stiglerEconomicsInformation1961}, CBA is straightforward in principle: the only hurdle that needs to be checked is that estimated information benefits outweigh (often low) implementation costs. But recent research (reviewed by \citet{golmanInformationAvoidance2017}) has challenged the traditional assumption, noting that information can evoke negative feelings such as anxiety or guilt, so that even costless information can harm people (i.e., the value of information can be negative).\footnote{For example, if information evoke negative emotions or feelings (e.g., anxiety, fear, guilt, regret, annoyance, or pain), these negative emotions or feelings might lead to information avoidance \citep{koszegiHealthAnxietyPatient2003,danaExploitingMoralWiggle2007,karlssonOstrichEffectSelective2009,sweenyInformationAvoidanceWho2010,osterOptimalExpectationsLimited2013,grossmanStrategicIgnoranceRobustness2014,hertwigHomoIgnoransDeliberately2016,onwezenWhenIndifferenceAmbivalence2016,savolainenApproachingAffectiveBarriers2016,thunstromStrategicSelfignorance2016,grossmanSelfImageWillfulIgnorance2017,damgaardHiddenCostsNudging2018,thunstromEndogenousAttentionCosts2019,sunsteinRuiningPopcornWelfare2019}.} The possibility to do harm with information policies complicates welfare analysis because\textemdash even ignoring the cost of implementation\textemdash their welfare effects can range from negative to positive \citep{damgaardHiddenCostsNudging2018, allcottWelfareEffectsNudges2019,buteraDeadweightLossSocial2019,rafiqHowMuchCalorieundefined/ed,thunstromWelfareEffectsNudges2020}.\footnote{The complication is greater if there is heterogeneity in negative feelings \citep{sunsteinRuiningPopcornWelfare2019}, or in the ability to respond to information \citep{robertsNudgeProofDistributiveJustice2018}}

Our findings point to further complications. Most immediately, the existence of an endowment effect for information implies that the outcome of welfare analysis may depend on timing. Consider, for example, a cost-benefit analysis of a policy that requires cafes to provide calorie information on their menus. In the presence of experience effects like those identified in our experiment, this analysis would be more likely to conclude in favor of the policy if conducted after the policy’s introduction, when people have come to expect the information. More generally, the existence of an endowment effect for information implies that the role of the referent, including potential heterogeneity of referents across consumers, ought to be explicitly acknowledged and accounted for in welfare analysis.

     % XXXXXXXX    PARAGRAPH BELOW NEEDS SOME WORK     XXXXXXXXXXXXXX %

On the positive side, the endowment effect for information may provide a new policy tool to reduce information avoidance when socially desirable (e.g., information about health and financial risks). The goal of this policy tool would be to influence people's expectation to receive information, and via the endowment effect, increase people's valuation of information (leading to less information avoidance). For example, during the COVID-19 pandemic, people might be less likely to ignore information about the risks of engaging in various activities if they expect to get such information.

We return to discussion of welfare implications in the concluding section, after first introducing our theoretical analysis and empirical evidence in the next sections.
