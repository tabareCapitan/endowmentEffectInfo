% PENDING
%
% - Change name of stata variables to something better
%   - Perhaps a table?

\subsection{Method}

We start with (22) potential covariates based on the answers to the survey questions at the end of the experiment.\footnote{We do not claim to have captured \emph{all} the characteristics that may be prognostic of information preferences, instead, we used our expertise to identify the ones we considered to be more important. As with any other empirical study, the rest will be captured in the error term of the econometric analyses.} Instead of (subjectively) assessing the degree of prognosis of each of these potential covariates, we assume that all of them are prognostic of the dependent variable\textemdash i.e., information preferences. Thus, the task simplifies to identify \emph{large} imbalances between groups.

We use an eclectic approach to identify \emph{large} differences, but our method closely follows advice by \citet{imbensCausalInferenceStatistics2015} to assess differences in the distribution of covariates between groups in the context of observational studies. For each covariate of a pair of groups (i.e., control and treatment), we calculate the univariate normalized difference in means, the multivariate difference in covariate distributions, and the natural logarithm of the ratio of standard deviations. A description of each estimator, taken from \citet{imbensCausalInferenceStatistics2015}, follows.

        % PENDING: CHECK BOOK TO VERIFY PAGES

\begin{displayquote}

Let $\bar{X_c}$ and $\bar{X_t}$ denote the sample averages of the covariate values for the control and treatment group respectively:

\begin{equation*}
  \bar{X_c}=\frac{1}{N_c} \sum_{i:W_i=0} X_i
  \text{  and  }
  \bar{X_t}=\frac{1}{N_t} \sum_{i:W_i=0} X_i,
\end{equation*}

where (\ldots) $N_c$ is the number of control units, and $N_t$ is the number of treated units. Also, let $s_c^2$ and $s_t^2$ denote the conditional within-group sample variances of the covariate:

\begin{equation*}
  s_c^2=\frac{1}{N_c-1} \sum_{i:W_i=0} (X_i-\bar{X_c})^2
  \text{  and  }
  s_t^2=\frac{1}{N_t-1} \sum_{i:W_i=0} (X_i-\bar{X_t})^2,
\end{equation*}

\end{displayquote}

Then the estimator of the normalized difference of means is

\begin{equation*}
  \hat{\Delta}_{ct}=
    \frac{\bar{X_t}-\bar{X_c}}
         {\sqrt{\frac{s_c^2+s_c^2}{2}}}
\end{equation*}

% CHECK REFERENCES BELOW

As a practical general rule, \citet{imbensCausalInferenceStatistics2015} say \enquote{\ldots when treatment groups have important covariates that are more than one-quarter or one-half of a standard deviation apart, simple regression methods are unreliable for removing biases associated with differences in covariates\ldots} (p.277).\footnote{First, we note that this practical general rule is given in the context of observational studies. Second, we note that there are at least three versions of the normalized differences in means estimator in work by Guido Imbens and they all share the same threshold (i.e., differences larger than 0.25 standard deviations are large). We use the version in \citet{imbensCausalInferenceStatistics2015} because it is the most recent. In Imbens \& Woolridge (2009, pp 24, JEL), the estimator sums up the variances but does not divide by 2. In Imbens \& Woolridge (NBER summer, 2007, p.34) and Imbens \& Wooldrige (Cemmep, UCL, 2000), the denominator is only the sum of the variances. Because different estimators share the same practical general rule (although the last version does include two thresholds) along with the fact that we have not found supporting evidence behind the $0.25$ threshold, we take the practical general rule as expert advice but do not rely solely on this rule to conclude.} \citet{imbensCausalInferenceStatistics2015} also suggest to use the multivariate difference in covariate distributions (p.313), however, there is no clear guidance on how large a difference should be to be considered \emph{large}.\footnote{In their illustrations on chapter 14, \citet{imbensCausalInferenceStatistics2015} interpret some of the multivariate difference in distribution estimates. In page 320, they interpret an estimate of 1.78 as suggestive that \enquote{\ldots overall the two groups are substantially apart}. In page 353, a table including an estimate of 1.49 is discussed and interpreted as evidence of substantial differences, but the multivariate differences in distribution estimator is not interpreted directly. In page 324, a table including an estimate of 0.44 is discussed and interpreted as \enquote{suggesting that the balance in covariates is excellent\ldots}, but it does not interpret the estimator directly. Following this interpretation, one could consider differences lower or equal than 0.44 as small and differences higher or equal to 1.49 as large.}

We also compare measures of dispersion of the covariates between groups. Using the same notation from \citet{imbensCausalInferenceStatistics2015}, the estimator is the natural logarithm ratio of standard deviations
\begin{equation*}
  \hat{\Gamma}_{ct}=ln(s_t)-ln(s_c).
\end{equation*}
We did not find any practical guidance to assess whether a difference is large or not.

These measures\textemdash univariate and multivariate difference in means and logarithm ratio of standard deviations\textemdash give us a way to compare distributions based on the first two moments of their distributions. However, that is not enough to capture imbalances that would matter if a covariate is prognostic. Instead of inspecting normalized differences for higher-order moments of the distributions, as suggested in page 313 by \citet{imbensCausalInferenceStatistics2015}, we take their complementary advice:

\begin{displayquote}
  Finally, it may be useful to construct histograms of the distribution of a covariate in both treatment arms to detect visually subtle differences not captured by differences in means and variances, especially for covariates that are a priori believed to be highly associated with the outcomes.
\end{displayquote}

Consequently, for each potential covariate, we use the results of the comparison of the normalized differences of means and variances, and the visual inspection of the histograms (or kernel densities for continuous variables), to identify \emph{large} differences.

\subsection{Results}

Here, we list the covariates with \emph{large} imbalances between the treatment and control group identified for each of the three relevant pairs of groups. In the online supplementary material we include a detailed description of how the application of the method for each pair of groups leads to the results presented in this appendix.

\begin{enumerate}
  \item The group of participants who received menus without calorie information in their hypothetical choices (low baseline, groups A and B) and the group of participants who received menus with calorie information in their hypothetical choices (high baseline, groups C and D).

  There is only one covariate with a \emph{large} imbalance:
  \begin{itemize}
    \item Risk preferences (higher in the low baseline)
  \end{itemize}

  \item Within participants who received a menu without calorie information for their hypothetical choices (low baseline, groups A and B): The group if participants who received a menu without calorie information for their real choice (group A) and the group of participants who received a menu with covered calorie information for their real choice (group B).

  There are eight covariates with a \emph{large} imbalance:
  \begin{itemize}
    \item Expenses level (higher in control, group A)
    \item Risk preferences (higher in control, group A)
    \item Present bias ($\beta$) (higher in control, group A)
    \item Experience with calorie information (higher in control, group A)
    \item College (higher in treatment, group B)
    \item Health assessment (higher in treatment, group B)
    \item Wish could eat healthier out (higher in treatment, group B)
    \item Knows calorie needs (higher in treatment, group B)
  \end{itemize}

  \item Within participants who received a menu with calorie information for their hypothetical choices (high baseline, groups A and B): The group of participants who received a menu without calorie information for their real choice (group C) and the group of participants who received a menu with covered calorie information for their real choice (group D).

  There are nine covariates with a \emph{large} imbalance:
  \begin{itemize}
    \item Importance of exercising regularly (higher in control, group C)
    \item Importance of healthy body weight (higher in control, group C)
    \item Knows calorie needs (higher in control, group C)
    \item Age (higher in treatment, group D)
    \item Hunger level (higher in treatment, group D)
    \item Health assessment (higher in treatment, group D)
    \item Frequency visits to chain restaurants (higher in treatment, group D)
    \item Expenses level (unclear)
    \item Experience with calorie information (unclear)
  \end{itemize}

\end{enumerate}
