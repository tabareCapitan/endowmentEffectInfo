
\title{\vspace{-3cm}
      Expecting to get it: \\ An endowment effect for information
      }


\author{Anonymous
        }

\maketitle

\thispagestyle{empty}   % Suppress page number (must be under \maketitle)

\begin{abstract}

\noindent
In this paper we predict and find evidence for an \enquote{endowment effect for information}---a tendency to value information more if getting the information is expected than if it is not expected. We show that the two leading theories of expectations-based reference-dependent preferences imply such an endowment effect, and find evidence supporting this prediction in an experiment that manipulates participants’ expectations. The effect implies that the net benefits from information policies may vary with people’s expectations: consumers who regularly see information about the calories in their food, the energy use of their appliances, or the carbon footprint of their flights may come to expect access to such information and then end up valuing it more.
\\
\\
\textit{Keywords:} expectations, reference-dependent preferences, preferences for information, information avoidance, endowment effect, welfare analysis.
\\
\textit{JEL classification:} D01, D80, D83, D84, D91, C91.

\end{abstract}

\clearpage

\pagenumbering{arabic} % Start numbering in page 2 from #1
