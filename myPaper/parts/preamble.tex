% PACKAGES ---------------------------------------------------------------------
\setcounter{secnumdepth}{1}
\usepackage[utf8]{inputenc}             % To write accents
\usepackage[english]{babel}             % So LaTeX do proper hyphenation
\usepackage[T1]{fontenc}                % Nicer default font (+ math font)
\usepackage{csquotes}                   % To use \enquote{}
\usepackage{titletoc}                   % Table of contents
                                        % -> must load before hyperref
\usepackage{hyperref}                   % Enable links
\usepackage[margin=1in]{geometry}       % To make pretty tables
\usepackage{booktabs}                   % To make pretty tables
\usepackage{tabularx}                   % To make pretty tables
\usepackage{mathtools}                  % Expand math tools
\usepackage{amsfonts}                   % Expand math tools
\usepackage{amsmath}                    % Expand math tools
\usepackage{amssymb}                    % Expand math tools
\usepackage{mathpazo}                   % Expand math tools
\usepackage{mathabx}                    % Expand math tools
\usepackage{graphicx}                   % Control figures
\usepackage[small,bf,center]{caption}   % Prettier figures
\usepackage{subfig}                     % Prettier figures
\usepackage{setspace}                   % ...
\usepackage{float}                      % To control figures' placement
\usepackage[section]{placeins}          % To control figures' placement
                                        % with \FloatBarrier
\usepackage{indentfirst}                % I like consistency!
\usepackage[toc,page]{appendix}         % To add appendices
\usepackage[authoryear]{natbib}         % References
  \bibliographystyle{chicago}           % -> style

% MATH SHORTCUTS ---------------------------------------------------------------

            % PENDING CLEAN UNUSED ONES !!!!!!!!!!!!!!!!!!!
\setcounter{secnumdepth}{0}
\let\IG\iffalse
\let\ENDIG\fi
\def\D{\displaystyle}
\def\td#1#2{\frac{\D \mathit{d} #1}{\D \mathit{d} #2}}
\def\std#1#2{\frac{\D \mathit{d}^2 #1}{\D \mathit{d} {#2}^2}}
\def\ctd#1#2#3{\frac{\D \mathit{d}^2 #1}{\D \mathit{d} #2 \mathit{d} #3}}
\def\pd#1#2{\frac{\D \partial #1}{\D \partial #2}}
\def\pdi#1#2{\partial #1/\partial #2}
\def\cpdi#1#2#3{\partial^2 #1/\partial #2 \partial #3}
\def\spdi#1#2{\partial^2 #1/\partial {#2}^2}
\def\spd#1#2{\frac{\D \partial^2 #1}{\D \partial {#2}^2}}
\def\cpd#1#2#3{\frac{\D \partial^2 #1}{\D \partial #2 \partial #3}}
\newcommand{\Lg}{\mathcal{L}}
\newcommand{\LR}{\Leftrightarrow}
\newcommand{\half}{\tfrac{1}{2}}
\newcommand{\qrtr}{\tfrac{1}{4}}
\newcommand{\eqs}{\buildrel s \over =}
\newcommand{\foc}[1]{\ensuremath{\text{foc}\mspace{2mu}#1}}
\newcommand{\cs}[1]{\ensuremath{\text{cs}\mspace{2mu}#1}}
\newcommand{\nn}[1]{\ensuremath{\text{nn}\mspace{2mu}#1}}
\allowdisplaybreaks

\newcommand{\brk}{\vspace*{0.8em}\hrule}


\newtheorem{prop}{Proposition}
