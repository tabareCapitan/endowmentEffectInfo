% SUMMARY

The general topic of this paper is how people value information; and the key insight is that information preferences are reference-dependent. The ex-ante value assigned to a particular piece of information depends on, at least, one prior belief (i.e., the referent) about the delivery (do I expect to get it?), content (do I expect bad news?), or usefulness (do I expect to change anything in response?) of that piece of information. Further, due to the incorporeal nature of information, all possible referents are expectations-based. In this paper we focus on the case in which the referent is a prior belief about the delivery of information.

More precisely, in this paper we predict and find evidence for an \enquote{endowment effect for information}, which is a tendency to value information more if getting the information is expected than if it is not expected. The two leading theories of expectations-based reference-dependent preferences---the Disappointment Aversion and the Kőszegi and Rabin approach---imply this effect when information is instrumental, but only the Kőszegi and Rabin approach predicts this effect when information is not instrumental. Further, we find evidence supporting the prediction of an endowment effect for information in a laboratory experiment that manipulates participants’ expectations to receive instrumental information.

% CBA implications

The existence of this endowment effect for information complicates welfare analysis because the effect implies that net benefits of information policies may vary with people’s expectations. In other words, consumers who regularly see information about the calories in their food, the risks of drunk driving, or the benefits of exercising may come to expect access to such information and then end up valuing it more.\footnote{For example, to gain support for an information policy, politicians may therefore choose to first implement the policy as a trial with a clear end date, with the intent to reevaluate the policy at that point (the public might be more willing to accept a trial of a policy). Such increase in public support after a trial period has been found for other policies---\citet{cherryImpactTrialRuns2014} show that public acceptance of a congestion tax increased after a trial period with the tax.} As such, the existence of an endowment effect for information implies that the role of the referent, including potential heterogeneity of referents across consumers, ought to be explicitly acknowledged and accounted for in welfare analysis.

On the other hand, the same implication that the net benefits of information policies may vary with people’s expectations points towards a potential new policy tool to reduce the tendency to avoid \enquote{good} information (e.g., information about health risk). Beyond policies to provide information, the new policy tool would focus on raising people’s expectation to receive information. In turn, via the endowment effect, this expectation would increase people’s valuation of information (which leads to less information avoidance).\footnote{As this new policy tool aims to manipulate expectations, the process by which people form expectations would need to be better understood. \possessivecitet{heffetzAreReferencePoints2018} refinement of \possessivecitet{koszegiModelReferenceDependentPreferences2006} definition of the referent is a step in the right direction (i.e., the referent is lagged beliefs that have \enquote{sunk-in}, instead of just lagged beliefs). Future research may explore whether referent spill-overs across domains of information occur: does having information in one domain increase people’s expectations of getting information in general?}


\clearpage
