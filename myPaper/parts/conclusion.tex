
% START: NEEDS REFINEMENT \\\\\\\\\\\\\\\\\\\\\\\\\\\\\\\\\\\\\\\\\\\\\\\\\\\\\\

In this paper we use two complementary approaches to introduce a new concept: the endowment effect for information. First, we show that two leading theories of expectations-based reference-dependent preferences (i.e., the Disappointment Aversion and the Kőszegi and Rabin approach) predict an endowment effect for instrumental information, but only the Kőszegi and Rabin approach predicts this effect when information is not instrumental. Second, we find this endowment effect in a laboratory experiment in which we manipulate the expectations to get information.

This endowment effect for information has implications for welfare analysis of information policies (e.g., cost-benefit analysis); we raise the thorny issue of explicitly addressing the role of the referent.  Here, we provide a few insights to what this might mean to researchers and practitioners doing welfare analysis. First, the effect of the endowment effect for information in welfare analysis might be easier to identify if expectations are fairly homogenous. For example, one could assume that the result of a cost-benefit analysis based on benefits and costs elicited from a sample of people who (mostly) do not expect to receive information would be more positive had the people in the sample expected to receive information.  Second, a direct implication of the endowment effect for information is that popular support for information policies would increase once the policy is implemented. To gain support for an information policy, politicians may therefore favor to first implement the policy as a trial with a clear end date, with the intent to re-evaluate the policy at that point. The public might be more willing to accept a trial of a policy. Such increase in public support after a trial period has been found for other policies\textemdash \citet{cherryImpactTrialRuns2014} show that public acceptance of a congestion tax increased after a trial period with the tax. Third, welfare analysis of information policies might be influenced by the analysis itself. Questions underlying contingent valuation might create the expectation to receive information and increase the elicited value of a given policy.

Finally, the endowment effect could be exploited to encourage information uptake in contexts where increased uptake would increase social welfare, e.g., information on climate change or benefits of using masks during the COVID-19 global pandemic. Our results suggest that this new policy tool would have greater potential in settings in which most people do not expect to get information.  Further, as the policy aims to manipulate expectations, the process by which people form expectations needs to be better understood. \citet{heffetzAreReferencePoints2018}’s refinement of \citet{koszegiModelReferenceDependentPreferences2006}’s definition of the referent is a step in the right direction (i.e., the referent is lagged beliefs that sunk-in, instead of just lagged beliefs).

Future research may explore the robustness of the endowment effect across types of information and contexts (i.e., referent spill-overs across domains of information). In other words, does having information in one domain increase people’s expectations of getting information in general? For example, does access to calorie information increase support for information about environmental risks?

% END WORKING AREA \\\\\\\\\\\\\\\\\\\\\\\\\\\\\\\\\\\\\\\\\\\\\\\\\\\\\\\\\\\\\

\clearpage
