\subsection{Proofs}

All proofs are implied from \citet{koszegiReferenceDependentRiskAttitudes2007}.

\subsection{Derivations}

Under the DA approach, applying the general formula
\begin{equation*}
  U(L|R) \equiv \sum_{n=1}^N p_n\biggl\{u(x_n) + \mu\biggl(u(x_n) - \sum_{m=1}^M q_m
u(r_m)\biggr)\biggr\}
\end{equation*}
yields that
\begin{align*}
  U(L^i|R^i)
&= p_1\bigl\{x_1 + \mu\bigl(x_1 - \underbrace{[q_1r_1 + q_2r_2]}_{E[R^i]}\bigr)\bigr\}
 + p_2\bigl\{x_2 + \mu\bigl(x_2 - \underbrace{[q_1r_1 + q_2r_2]}_{E[R^i]}\bigr)\bigr\}
\\
&= \half\bigl\{h + \bigl(h - [\half h + \half\ell]\bigr)\bigr\}
 + \half\bigl\{\ell + \lambda\bigl(\ell - [\half h + \half \ell]\bigr)\bigr\}
\\
&= [\half h + \half \ell] + \half\bigl(h - [\half h + \half\ell]\bigr)
   - \half\lambda\bigl([\half h + \half \ell] - \ell\bigr)
\\
&= [\half h + \half \ell] - \qrtr(\lambda - 1)(h - \ell).
\end{align*}
If the person ends up getting information, then with probability $\half$ she
will learn that the cake is low-calorie. She will then get high intrinsic
utility $h$ from consuming it, and will in addition experience ``elation'' $h
- [\half h + \half\ell]$, because $h$ exceeds the intrinsic utility $\half h +
\half \ell$ that she expected to get---her reference utility. Also with
probability $\half$, however, the cake will turn out to be high-calorie. The
person will then get low intrinsic utility $\ell$ from consuming it, and will
in addition experience ``disappointment'' $\lambda(\ell - [\half h +
\half\ell])$, because $\ell$ falls short of her reference utility. If $\lambda
> 1$, the disappointment will outweigh the elation, leaving her with expected
intrinsic utility $\half h + \half \ell$ from consuming the cake, but in
addition net negative expected gain-loss utility $- \qrtr(\lambda - 1)(h -
\ell) < 0$.

The same formula yields that
\begin{align*}
  U(L^u|R^i)
&= p\bigl\{x + \mu\bigl(x - \underbrace{[q_1r_1 + q_2r_2]}_{E[R^i]}\bigr)\bigr\}
\\
&= 1\cdot\bigl\{[\half h + \half \ell] + \bigl([\half h + \half \ell] - [\half h + \half \ell]\bigr)\bigr\}\\
&= [\half h + \half \ell].
\end{align*}
Here, because the person ends up not getting information, she will get average
intrinsic utility $\half h + \half \ell$ for sure. And because this is exactly
what she expected to get, she experiences no elation or disappointment.

Comparing the two utilities yields that
\begin{equation*}
  U(L^i|R^i) - U(L^u|R^i) = -\qrtr(\lambda-1)(h-\ell) < 0.
\end{equation*}
This shows that, because the person is disappointment averse ($\lambda > 1$),
she will also be information adverse: she perceives herself as better off
staying ignorant, because her disappointment from getting bad news would
outweigh her elation from getting good news.

Crucially, this result holds regardless of whether she started out expecting
to get information or not, i.e., with reference lottery $R^i$ or $R^u$. This
follows because the DA approach ``collapses'' the reference lottery into its
expectation before evaluating gain-loss utility. And because $R^i$ and $R^u$
have the same expectation $E[R^i] = E[R^u] = q_1r_1 + q_2r_2$, we find that
$U(L^i|R^u) = U(L^i|R^i)$ and $U(L^u|R^u) = U(L^u|R^i)$, so also
\begin{equation*}
  U(L^i|R^u) - U(L^u|R^u) = U(L^i|R^i) - U(L^u|R^i) = -\qrtr(\lambda-1)(h-\ell) < 0.
\end{equation*}
There is, in other words no endowment effect for information: the person is
equally information adverse, regardless of whether she expects to get
information or not.

Under the KR approach, however, the situation is quite different. Applying the
general formula
\begin{equation*}
  U(L|R) \equiv \sum_{n=1}^N p_n\biggl\{u(x_n) + \sum_{m=1}^M q_m
\mu\bigl(u(x_n) - u(r_m)\bigr)\biggr\}
\end{equation*}
yields that
\begin{align*}
  U(L^i|R^i)
&= p_1\bigl\{x_1 + q_1\mu(x_1 - r_1) + q_2\mu(x_1 - r_2)\bigr\}
 + p_2\bigl\{x_2 + q_1\mu(x_2 - r_1) + q_2\mu(x_2 - r_2)\bigr\}\\
&= \half\bigl\{h + \half(h - h) + \half(h - \ell)\bigr\}
 + \half\bigl\{\ell + \half\lambda(\ell - h) + \half(\ell - \ell)\bigr\}\\
&= [\half h + \half \ell] +
    \qrtr (h-\ell) - \qrtr \lambda(h-\ell)\\
&= [\half h + \half \ell] - \qrtr(\lambda-1)(h - \ell).
\end{align*}
Again, if the person ends up getting information, she will learn with
probability $\half$ that the cake is low-calorie and then get high intrinsic
utility $h$ from consuming it. In addition, she will compare that utility {\em
separately} to the two outcomes she {\em expected} to potentially get, given
that she expected to be informed. Since she started out expecting to get $h$
with probability $\half$, she will experience no elation from that first
comparison; but since she expected to get only $\ell$ with probability $\half$
also, she will experience elation $h-\ell$ from that second comparison. Also
with probability $\half$, the cake will turn out to be high-calorie, yielding
low intrinsic utility $\ell$. Comparing that utility again separately to the
two outcomes $h$ or $\ell$ that she expected to get, she will experience
disappointment $\lambda(\ell - h) < 0$ from the first comparison, and no
disappointment from the second comparison. Aggregating over all these
permutions leaves her with expected intrinsic utility $\half h + \half \ell$
from consuming the cake, but in addition net negative expected gain-loss
utility $-\qrtr(\lambda - 1)(h - \ell) < 0$.

The same formula yields that
\begin{align*}
  U(L^u|R^i)
&= p\bigl\{x + q_1\mu(x - r_1) + q_2\mu(x - r_2)\bigr\}
\\
&= 1\cdot\bigl\{[\half h + \half \ell] + \half\lambda([\half h + \half \ell] -
h) + \half([\half h + \half \ell] - \ell)\bigr\}\\
&= [\half h + \half \ell] +
   \half\left([\half h + \half \ell] - \ell\right)
   -\half\lambda\left(h - [\half h + \half \ell]\right)\\
&= [\half h + \half \ell] - \qrtr(\lambda-1)(h - \ell).
\end{align*}
Here again, because the person ends up not getting information, she will get
average intrinsic utility $\half h + \half \ell$ for sure. In addition, she
will again compare that utility separately to the two outcomes she expected to
potentially get, given that she expected to be informed. Since she started out
expecting to get $h$ with probability $\half$, she will experience
disappointment $\lambda([\half h + \half \ell] - h) < 0$ from getting the
lower average utility instead; but since she expected to get only $\ell$ with
probability $\half$ also, she will experience elation $[\half h + \half\ell] -
\ell$ from that second comparison. Aggregating leaves her with expected
intrinsic utility $\half h + \half \ell$ from consuming the cake, but in
addition net negative expected gain-loss utility $-\qrtr(\lambda - 1)(h -
\ell) < 0$.

Comparing these first two utilities yields that
\begin{equation*}
  U(L^i|R^i) - U(L^u|R^i) = 0.
\end{equation*}
This shows that, despite being disappointment averse, the person is indifferent
to getting information or not. The reason becomes clear if we compare the
third lines of the derivations above:
\begin{align*}
  U(L^i|R^i) &= [\half h + \half \ell] +
  \qrtr (h-\ell) - \qrtr \lambda(h-\ell)\\
  U(L^u|R^i) &= [\half h + \half \ell] +
  \half\left([\half h + \half \ell] - \ell\right)
  -\half\lambda\left(h - [\half h + \half \ell]\right).
\end{align*}
Relative to becoming informed, staying ignorant cuts both the gain and loss
utilities in half (from $[\half h + \half \ell] - \ell$ to $h - \ell$, and
from $-\lambda(h - [\half h + \half \ell])$ to $-\lambda(h - \ell)$,
respectively). But because it simultaneously doubles the probability that
either the gain utility or the loss utility will be experienced (from $\half$
to $\qrtr$), the net effect is a wash.

Next, we have
\begin{align*}
  U(L^i|R^u)
&= p_1\bigl\{x_1 + q\mu(x_1 - r)\bigr\}
 + p_2\bigl\{x_2 + q\mu(x_2 - r)\bigr\}\\=
&= \half\bigl\{h + 1\cdot(h - [\half h + \half\ell])\bigr\}
 + \half\bigl\{\ell + 1\cdot\lambda(\ell - [\half h + \half\ell])\bigr\}\\
&= [\half h + \half \ell] +
\half(h-[\half h + \half \ell]) - \half \lambda([\half h + \half\ell] - \ell)\\
&= [\half h + \half \ell] - \qrtr(\lambda-1)(h - \ell).
\end{align*}
If the person did not expect to be informed but ends up getting information,
she will compare the intrinsic utility from the outcomes $h$ and $\ell$ to the
utility $\half h + \half \ell$ that she expected to get. The elation $h -
[\half h + \half \ell]$ with probability $\half$ will again be outweighed by
the disappointment $ - \half \lambda([\half h + \half\ell] - \ell)$ with
probability $\half$, resulting in net negative expected gain-loss utility
$-\qrtr(\lambda - 1)(h - \ell) < 0$.

Lastly, we have
\begin{align*}
  U(L^u|R^u)
&= p\bigl\{x + q\mu(x - r)\bigr\}
\\
&= 1\cdot\bigl\{[\half h + \half \ell] + 1\cdot([\half h + \half \ell] - [\half
h + \half \ell])\bigr\}\\
&= [\half h + \half \ell].
\end{align*}
Since the person gets exactly what she expects, namely no information for
sure, she experiences no gain-loss utility.

Comparing the second two utilities yields that
\begin{equation*}
  U(L^i|R^u) - U(L^u|R^u) = -\qrtr(\lambda-1)(h-\ell) < 0.
\end{equation*}
This shows that, if the person expects not to get information, she will be
information adverse. But since we found above that
\begin{equation*}
  U(L^i|R^i) - U(L^u|R^i) = 0,
\end{equation*}
i.e., that the person is indifferent about information if she {\em does} expect
to get it, we now do have an endowment effect for information: the value of
information is higher, namely zero instead of negative, for people who expect
to get it.

\brk

\begin{center}
\textbf{Instrumental information}
\end{center}

{\color{red}
[Yet to be done. Remember that it doesn't change anything substantive
for the KR approach, though: it just makes information more valuable, by the
same amount, regardless of the reference lottery, so there is still exactly the
same endowment effect.

For the DA approach, it gives rise to an endowment effect as well (whereas
there isn't one with non-instrumental information), but we're not
claiming to differentiate between those approaches in our experiment anyway.
]
}
