
% START: NEEDS REFINEMENT \\\\\\\\\\\\\\\\\\\\\\\\\\\\\\\\\\\\\\\\\\\\\\\\\\\\\\

In this paper we introduce the endowment effect for information using two complementary approaches. First, we show that two leading theories of expectations-based reference-dependent preferences (i.e., the Disappointment Aversion and the Kőszegi and Rabin approach) predict an endowment effect for instrumental information, but only the Kőszegi and Rabin approach predicts this effect when information is not instrumental. Second, we find the endowment effect in a laboratory experiment in which we manipulate the expectations to get information.

However, we note that our experimental results are contingent on the unverified manipulation of expectations (i.e., probabilistic beliefs). First, we assume that making eleven choices using menus with or without calorie information changes the probabilistic belief to expect to receive information. Second, given the probabilistic belief previously assumed to have been manipulated, we assume that presenting a menu with or without covered calorie information, along with other framing differences, would further change the probability belief to expect to receive information or not. We chose not to verify that the manipulations in fact had the intended effect in forming the expectation to receive information. One alternative would have been to ask participants whether they were expecting to receive information after each manipulation, however, we thought that such verification would have unintendedly changed the expectations. A second alternative would have been to use a different experimental design to manipulate the expectations. Following \citet{marzilliericsonExpectationsEndowmentsEvidence2011} \citet{heffetzEndowmentEffectExpectations2014}, one could manipulate the expectations, for example to expect to receive information, by telling participants that they have a 1\% chance to be able to choose whether to get the information or not, and a 99\% chance to receive the information.\footnote{ More recently, \citet{cerulli-harmsRandomizingEndowmentsExperimental2019} show that forced exchange is preferred than the allowable exchange used by \citet{marzilliericsonExpectationsEndowmentsEvidence2011} and \citet{heffetzEndowmentEffectExpectations2014}.} The advantage of this approach is that one can verify that participants formed the correct expectations with a quiz evaluating the understanding of the statement. However, we think that our approach is more appropriate for the context of information and closer to potential policy applications outside of the laboratory. Overall, as reflected by the sensitivity of results to changes in the experimental environment documented in the empirical literature of expectations-based reference-dependent models, we can only conclude that manipulating expectations is a challenging task in which trade-offs are inevitable.

With respect to the implications of our endowment effect for welfare analysis of information policies (e.g., cost-benefit analysis), we raise the thorny issue of explicitly addressing the role of the referent.  Here, we provide a few insights to what this might mean to researchers and practitioners doing welfare analysis. First, the effect of the endowment effect for information in the welfare analysis might be easier to identify if expectations are fairly homogenous. For example, one could assume that the result of a cost-benefit analysis based on benefits and costs elicited from a sample of people who (mostly) do not expect to receive information would be more positive had the people in the sample expected to receive information.  Second, a direct implication of the endowment effect for information is that popular support for information policies would increase once the policy is implemented. To gain support for an information policy, politicians may therefore favor to first implement the policy as a trial with a clear end date, with the intent to re-evaluate the policy at that point. The public might be more willing to accept a trial of a policy. Such increase in public support after a trial period has been found for other policies\textemdash \citet{cherryImpactTrialRuns2014} show that public acceptance of a congestion tax increased after a trial period with the tax. Third, welfare analysis of information policies might be influenced by the analysis itself. Questions underlying contingent valuation might create the expectation to receive information and increase the elicited value of a given policy.

Finally, the endowment effect could be exploited to encourage information uptake in contexts where increased uptake would increase social welfare, e.g., information on climate change or other environmental issues. Our results suggest that this new policy tool would have the greater potential in settings in which most people do not expect to get information.  Further, as the policy aims to manipulate expectations, the process by which people form expectations needs to be better understood. \citet{heffetzAreReferencePoints2018}’s refinement of \citet{koszegiModelReferenceDependentPreferences2006}’s definition of the referent is a step in the right direction (i.e., the referent is lagged beliefs that sunk-in, instead of just lagged beliefs).

Future research may explore the robustness of the endowment effect across types of information and contexts. Future research may also explore if there are referent spill-overs across domains of information. In other words, does having information in one domain increase people’s expectations of getting information in general? For example, does access to calorie information increase support for information about environmental risks?

% END WORKING AREA \\\\\\\\\\\\\\\\\\\\\\\\\\\\\\\\\\\\\\\\\\\\\\\\\\\\\\\\\\\\\

\clearpage
