We recruited 267 participants during the Spring of 2018; most of them students from the University of Wyoming. After excluding 1 student with dietary restrictions (e.g., vegan or lactose intolerance), 18 students who participated in a different experiment with the same researchers a few weeks before this experiment, and 29 students who reported they did not intend to eat the cake themselves (implying that calorie content would not matter to them), we ended up with 219 valid observations.

We completed the experiment in 13 sessions with about 20 participants each; all of them with the same monitor in the same experimental laboratory of the College of Business in the University of Wyoming.\footnote{Aiming to \emph{normalize} the hunger level of participants during the experiment, we conducted all sessions either before lunch (11 AM) or before dinner (5 PM)} Upon arrival to the laboratory, we assigned participants to one of the computers and asked them not to communicate with each other or use their cell phones during the experiment. The session started with a short lecture with examples on how to use a multiple-price list.\footnote{The slides used for the lecture are included in the online supplementary material} After that, each participant started the experiment on their own. The experiment, including randomizations, were done using Qualtrics. At the end of the experiment, we paid each participant in a private room. They received a \$20 show-up fee plus \$1.46 on average for their earnings in the multiple-price list. Finally, we gave each participant one dessert that we bought at Walmart.\footnote{We covered the information about the calorie content in the nutritional label with a marker before the experiment started.} On average, the experiment took about 35 minutes.
